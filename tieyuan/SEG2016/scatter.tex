
\title{Data-driven diffraction imaging using passive seismic data}
\renewcommand{\thefootnote}{\fnsymbol{footnote}}

\author{Tieyuan Zhu\footnotemark[1]\footnotemark[2], Junzhe Sun\footnotemark[2], Sergey Fomel \footnotemark[2],\\ \footnotemark[1]Pennsylvania State University, \footnotemark[2]The University of Texas at Austin}
\email{tieyuanzhu@gmail.com}

%\address{
%\footnotemark[1]Department of Geosciences, Pennsylvania State University, State College, PA 16708.
%        Email: tieyuanzhu@gmail.com  \\
%\footnotemark[2]Jackson School of Geosciences, The University of Texas at Austin, Austin, TX 78758. \\
%        }

\maketitle

%-----------------------------------------------------------------
\begin{abstract}
Diffraction imaging tends to provide additional high-resolution subsurface imaging. This paper present a diffraction imaging without the need of source, in particular for passive seismic data in which source information usually is not available or inaccurate.
Nature fractures are desirable to be mapped for its effects on the further stimulation volume when hydraulic fracturing shale formation. Traditional three-dimensional seismic imaging for mapping such targets usually needs to generate source wavefields from known source locations. This may be difficult to apply to microseismic data that is most likely to be uncertain of source location. This paper presents a complimentary imaging approach that uses microseismic coda to image nature fractures without knowledge of source. Transmitted and scattered wavefields are back propagated from known receiver arrays (surface and/or borehole). The imaging formula describes that direct waves coincide with scatter waves at scatter points. The transmitted and scattered wave fronts are in the same location only at the time of conversion. Both P- and S-waves are considered. We also evaluate surface and/or borehole acquisition scenarios using synthetic examples. Results suggest that using S coda waves give better resolution imaging of natural fractures.
\end{abstract}

%-----------------------------------------------------------------
\section{Introduction}
Natural fractures are important for refracturing (Yang and Zoback, 2015). Coda waves are generated by scatterers, faults, fractures/cracks when seismic waves pass through these bodies. 

Microseismic monitoring using three-component sensors often contains clear P-wave and S-wave direct wave arrivals, which are picked (mainly traveltimes) to characterize microseismic source, e.g., location, mechanisms), and further infer fracture network (). Full wavefield data contains more fruitful information, e.g., P and S waveforms, reflections, conversions, scattered waves. It is believed that using full wavefields with less reliance on phase picking is promising to provide a more complete and reliable real-time solution. Recent studies have shown that using waveform information could give better constraints of microseismic sources thus induced fractures (Duncun et al., 2008; Song et al., 2010). In addition, the frequency content of microseismic data is often high, 200 Hz – 1000 Hz, which implies the ability to imaging small-scale structures. This paper presents a method to use microseismic coda waves that are very significant in microseismic data to imaging large-scale fractures. 

Three-dimensional seismic imaging is a powerful technique to map faults and fractures. For example, diffraction imaging (Fomel et al., 2007; Luker et al., 2015; Popovici et al., 2015). Concurrently, the wavefield from the source is forward propagated into the medium. An image of the scatterers is formed from the product of the source-forward propagated and receiver-backpropagated wavefields at each time step (a process that extracts the time- and space-coincident forward and back propagated wavefields) and adding the current image to the images obtained at earlier time steps. Microseismic source characterization (e.g., location and radiation pattern) is complicated and difficult to estimate in practice. The big issue of this data is to determine the source location. Since the source wavefield is extrapolated from source. The uncertainties of source location must bring error in seismic image. A basic limitation of this approach for microseismic imaging is that it requires knowledge of the source location. 

In past studies, imaging of subsurface heterogeneities with passive seismic waves was attempted by either assuming that each point in a three-dimensional volume beneath the array is a point scatter, computing the moveout trajectory for the scattering point, and then computing the semblance over time along this trajectory in the data (Cole, 1995), or by backpropagating the recorded wavefield back into the medium using an integral representation and looking for focusing at each depth as the scattered waves converge on the primary wave (Troitskiy et al., 1981; Potter, 1994; Norton and Won, 2000). Both of these methods differ from conventional imaging using active sources in that they lack an imaging condition that tells the algorithm how to refocus the scattered waves at the location in the medium where they were generated. Nihei et al. (1999) first presented the idea of using backpropagation of the directed and converted P-S waves in the reverse-time migration of VSP data. They implemented the elastic reverse-time migration for fracture imaging. Xiao et al. (2010), 
I propose a modified reverse-time migration and applied it to microseismic data, which constructs depth images with no knowledge of the source and operates on the original seismograms. Its main feature is an approximation of the surface source wavefield in the target area using the transmitted P-wave recorded in a surface or borehole array, by its reverse-time extrapolation. We have shown that this procedure gives a ‘kinematically’ reliable approximation of the source wavefield in a region near the well where a direct P-wave, recorded in the well, passed through. This fact causes the main limitation of the suggested approach.


The paper is organized as follows. We begin by illustrating a poor illumination problem caused by high attenuation zones when applying elastic RTM using a synthetic model. Next we describe the theory of viscoelastic RTM by reviewing viscoelastic wave equation, formulating viscoelastic reverse-time propagation with attenuation compensation, and describing the decomposed elastic imaging conditions in viscoelastic media. Finally we demonstrate the effectiveness of the algorithm using two synthetic models.

%-----------------------------------------------------------------
\section{Methodology}
Conventionally, the image principle states that reflectors exist at points in the ground where the first arrival of downgoing wave is time coincide with an upgoing wave (Claerbout, 1971). Based on this, reverse-time migration produces an image of the subsurface by extrapolating both the source downgoing and receiver upgoing wavefields into the interior of the earth. The imaging condition (Claerbout, 1971) consists of crosscorrelating the two wavefields at each time step. Reflectors are formed where the two wavefields correlate. 
Analogously, I describe a reverse-time imaging methodology that is based on reverse time backpropagation of direct and scattered wavefields recorded along a receiver array, instead of forward propagating a source wavefield. 

The data consists of primary waves  $T(x,t)$ and scattered waves $S(x,t)$,
\begin{equation}
\label{eq:eq1}                      
\hspace{0pt} D(x,t) = T(x,t) + S(x,t)
\end{equation}

Figure 1 shows this principle is also true for transmission wavefield., the coming transmission wave interact with scatters that produce scattered wavefields accompanying with transmission wavefield until scatters disappear. We restate the image condition in the transmission sense: the transmitted and scattered wave fronts are in the same location only at the time of conversion. 
 (1). Separate the direct waves (P- and/or S-wave) and its scattered waves (here we define the scattered waves here as the part of the wavefield arriving after the direct waves that includes P-P and P-S waves) into two wavefields, 
(2). Using a priori information about the smooth background (P- and/or S-wave) velocity structure (e.g., obtained from travel time tomography), time reverse the two multicomponent wavefields recorded along the surface sensor array such that first data in becomes the last data out. 
(3). Back propagation of a transmitted wavefield at the receiver   location through an appropriate earth model (the transmitted wavefield  ).
(4). Back propagation of a scattered wavefield at the receiver location   through the same model (the scattered wavefield  ).
(5). Applying the zero-lag crosscorrelation of the backpropagated direct and scattered wavefields to form images of heterogeneities at each grid location in the finite difference model that produce scattered waves (e.g., fractures, layer interfaces).
 
We repeat steps 1-5 for each time step in the backpropagation of the wavefields from the receiver array into the model domain. The image of the fractures and other heterogeneities that generate scattered waves is produced by summing the result of step 5 for all the time steps required to backpropagate the wavefield from the receiver array through the imaging region. It is obvious that at points in space where there is no scatter, the product of   and  will vanish. When the deconvolution condition is applied, the value of the image  represents the transmission coefficient at that point.
The reverse time converted wave imaging scheme for passive seismic waves does not require any knowledge of the source properties, such as location, radiation pattern, or source time function.
 

%-----------------------------------------------------------------
\section{Numerical examples}
We present two examples with the need of detection of faults and fractures. The first example uses a borehole passive monitoring geometry to imaging the fractures. The second uses a surface geometry to delineate natural fractures.

\subsection{Borehole monitoring}
We adopt a P-wave velocity model in the San Andreas fault observatory at depth (SAFOD) project for monitoring micro-earthquakes.

\plot{safod/model}{width=0.4\textwidth}{P modes of vertical velocity component. Left panel: forward source wavefield. Middle panel: backward receiver wavefield. Right panel: crosscorrelated PP image. These are simulated by using (a) viscoelastic with attenuation compensation (b) elastic without compensation.}

\plot{safod/image}{width=0.4\textwidth}{P modes of vertical velocity component. Left panel: forward source wavefield. Middle panel: backward receiver wavefield. Right panel: crosscorrelated PP image. These are simulated by using (a) viscoelastic with attenuation compensation (b) elastic without compensation.}

error in source location
\plot{safod/error_souloc}{width=0.4\textwidth}{P modes of vertical velocity component. Left panel: forward source wavefield. Middle panel: backward receiver wavefield. Right panel: crosscorrelated PP image. These are simulated by using (a) viscoelastic with attenuation compensation (b) elastic without compensation.}
\subsection{Surface monitoring}

\section{Field example}

%-----------------------------------------------------------------
\section{Conclusions}
We have presented a theory of viscoelastic RTM. Synthetic examples have shown that it is possible to compensate for both P- and S-wave attenuation effects (i.e., amplitude loss and distorted phase) to better image the subsurface geologic structure. Thus viscoelastic RTM can be a powerful tool that provides a higher resolution and a amplitude-balanced illumination of target structures using multicomponent data. Success of viscoelastic RTM relies on two properties of the fundamental viscoelastic wave equation: 1) either amplitude loss or velocity dispersion can be independently enabled during wave propagation; 2) the back propagation equation is formulated by reversing the sign of both P- and S-wave loss operators, which simultaneously compensates for both P- and S-wave attenuation effects in the reconstructed waveforms. We believe that the proposed methodology opens the door for industrial viscoelastic reverse-time migration, least-squares RTM, and viscoelastic full-waveform inversion. 

%-----------------------------------------------------------------
\section{Acknowledgments}
T. Zhu was supported by the Jackson Postdoctoral Fellowship at the University of Texas at Austin and the startup funding from Department of Geosciences and Institute of Natural Gas Research at the Pennsylvania State University. J. Sun was supported by Statoil Fellowship at the University of Texas at Austin. 

\onecolumn

\bibliographystyle{seg}
\bibliography{ref032015}

